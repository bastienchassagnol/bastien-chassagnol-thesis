\chapter{Model biological networks}

\paragraph{Motivation about integrative modelling}
Instead of considering each gene as independent and static, the analyses reported in this chapter aim at recovering the holistic and dynamic structure of the transcriptome.
\todo{refer to section 1.2 in biology}

\section{Introduction to computational models in Systems Biology}

\subsection{The main paradigms in the study Biological Networks}

\subsubsection{Build networks}
\paragraph{Top-down vs. Bottom-up}
\paragraph{Knowledge vs. Pure Learning approaches}
\todo[fancyline]{refer to the Boolean network internship}
\paragraph{Model graphs}

\subsubsection{Static networks}

\subsubsection{Dynamic networks}

\begin{enumerate}[label=\alph*)]
\item \textbf{Boolean networks:}
\item \textbf{Random networks:}
\item \textbf{Mechanistic networks:}

\end{enumerate}



\section{Random networks}
\paragraph{Overview: model probalistic distributions in a compact manner}
\todo{resources from my internship and general introduction to ggm}
\subsection{Exploit conditional independence to generate compact and informative networks}
\subsection{Gaussian Graphical Networks}
\paragraph{Gaussian Graphical Networks in high-dimensional setting}
\subsection{Gaussian Bayesian Networks}
\subsection{From GGMs to GBNs}
\paragraph{Learn a \gls{gbn} under topological constraints}
\paragraph{Model selection}
\todo[fancyline]{\cite{dahl_etal05}}


\section{Differential network analysis}

\paragraph{Overview: dynamic identification of key biological drivers}
Predict dynamically the behaviour of a new drug and prioritise biomarkers integrating the graph structure
\subsection{A local-score approach for the identification of key biological drivers}




