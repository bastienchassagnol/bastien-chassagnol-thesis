\chapter{Mixture models} 
\label{chap:gmm-benchmark}

\paragraph{Overview: variability in biological systems}

In general in absence of clear identified phenotype marker, it is common to assume the same generative model for patients belonging to a given cohort. 
However, we observe strong variability even between related samples, included within the same individuals. The variability may result from three distinct factors: at the environmental level (disease state, tissue location, \ldots), the genotypical level (presence of individual mutations inducing varying transcriptomic activity) and even at the cell population level. 

When unobserved, using a \textit{latent variable} to account for the intra-variability may reveal useful to discover unknown patterns. In the next section, we present one of the most commonly used statistic tool, namely \Glspl{gmm}, used to perform \textit{unsupervised} analysis. 


\section{Article 1: Gaussian Mixture Models in R}

\includepdf[pages=-]{gaussian_mixtures_benchmark.pdf}











